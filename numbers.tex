% !TEX root = main.tex

\documentclass[main.tex]{subfiles}
\begin{document}
\chapter{Printing numbers}
So far we had seen a few software interrupt facilities of MSDOS:

\begin{tabular}{ll}
\toprule
Interrupt & What does it do? \\
\midrule
20 & Finishes the current program and clean registers \\
21 & It calls a special MSDOS function (setted in \texttt{AH}) \\
\bottomrule
\end{tabular}

And so far we had seen two important MSDOS \texttt{INT 21} functions:

\begin{tabular}{ll}
\toprule
Value in \texttt{AH} & MSDOS Function \\
\midrule
02 & Display in the screen whatever ASCII character is set in \texttt{DL} \\
09 & Print in the screen whatever text is at address specified at \texttt{DX} and ended in \texttt{\$} \\
\bottomrule
\end{tabular}

A simple program to print the letter \texttt{A} (ASCII hex code \texttt{0x41}) and then finish the program we could do something like this:

\begin{minted}{asm}
    MOV AH,02
    MOV DL,41
    INT 21
    INT 20
\end{minted}

Now, could we just put the number \texttt{8} in \texttt{DL} and print it to the screen? Well, not really, because what the MSDOS expects is the ASCII code and, well, the ASCII code for 8 is \texttt{0x38}. The numbers in ASCII start on \texttt{0x30} so for printing the number 8 we can add \texttt{0x30}:

\begin{minted}{asm}
    MOV AH,02
    MOV DL,08
    ADD DL,30
    INT 21
    INT 20
\end{minted}
\end{document}